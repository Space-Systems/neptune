%============================================================
%
%	Macros
%
%-----------------------------------
\makeatletter
\def\getnodedistance{\tikz@node@distance}
\makeatother

\makeatletter
\tikzset{
    getdist/.style args={#1 and #2}{
    getdistc={#1}{#2},minimum width=\mylength-\pgflinewidth
    },
getdistc/.code 2 args={
\pgfextra{
    \pgfpointdiff{\pgfpointanchor{#1}{west}}{\pgfpointanchor{#2}{east}}
    \xdef\mylength{\the\pgf@x}
         }
    }
}
\makeatother

% Let groupplots have one (centered) x and y label
% ----------------------------------------------------
\makeatletter
\pgfplotsset{
    groupplot xlabel/.initial={},
    every groupplot x label/.style={
        at={($({\pgfplots@group@name\space c1r\pgfplots@group@rows.west}|-{\pgfplots@group@name\space c1r\pgfplots@group@rows.outer south})!0.5!({\pgfplots@group@name\space c\pgfplots@group@columns r\pgfplots@group@rows.east}|-{\pgfplots@group@name\space c\pgfplots@group@columns r\pgfplots@group@rows.outer south})$)},
        anchor=north,
    },
    groupplot ylabel/.initial={},
    every groupplot y label/.style={
            rotate=90,
        at={($({\pgfplots@group@name\space c1r1.north}-|{\pgfplots@group@name\space c1r1.outer
west})!0.5!({\pgfplots@group@name\space c1r\pgfplots@group@rows.south}-|{\pgfplots@group@name\space c1r\pgfplots@group@rows.outer west})$)},
        anchor=south
    },
    execute at end groupplot/.code={%
      \node [/pgfplots/every groupplot x label]
{\pgfkeysvalueof{/pgfplots/groupplot xlabel}};  
      \node [/pgfplots/every groupplot y label] 
{\pgfkeysvalueof{/pgfplots/groupplot ylabel}};  
    }
}

\def\endpgfplots@environment@groupplot{%
    \endpgfplots@environment@opt%
    \pgfkeys{/pgfplots/execute at end groupplot}%
    \endgroup%
}
\makeatother
% ----------------------------------------------------


\newcommand*{\mytextstyle}{\bfseries\color{gray!90}}
\newcommand{\arcarrow}[8]{%
% inner radius, middle radius, outer radius, start angle,
% end angle, tip protusion angle, options, text
  \pgfmathsetmacro{\rin}{#1}
  \pgfmathsetmacro{\rmid}{#2}
  \pgfmathsetmacro{\rout}{#3}
  \pgfmathsetmacro{\astart}{#4}
  \pgfmathsetmacro{\aend}{#5}
  \pgfmathsetmacro{\atip}{#6}
  \fill[#7, drop shadow] (\astart:\rin) arc (\astart:\aend:\rin)
       -- (\aend+\atip:\rmid) -- (\aend:\rout) arc (\aend:\astart:\rout)
       -- (\astart+\atip:\rmid) -- cycle;
  \path[font = \sffamily, decoration = {text along path, text = {|\mytextstyle|#8},
    text align = {align = center}, raise = -0.5ex}, decorate]
    (\astart+\atip:\rmid) arc (\astart+\atip:\aend+\atip:\rmid);
}


%=======================================================
%
% 	Colors
%
%----------------------------------
\definecolor{ilr-blue}{RGB}{44,79,162}
\definecolor{light-blue}{RGB}{220,220,255}
%\definecolor{light-red}{RGB}{255,220,220}
\definecolor{gray90}{gray}{0.9}
\definecolor{gray80}{gray}{0.8}
\definecolor{gray60}{gray}{0.6}
\definecolor{gray50}{gray}{0.5}
\definecolor{gray40}{gray}{0.4}
\definecolor{gray20}{gray}{0.2}
\definecolor{mintedbg}{rgb}{0.93,0.93,0.93}

\colorlet{light-green}{green!20!white}
\colorlet{very-light-green}{green!5!white}
\colorlet{dark-green}{green!40!black}

\colorlet{light-yellow}{yellow!20!white}
\colorlet{very-light-yellow}{yellow!5!white}
\colorlet{dark-yellow}{yellow!40!black}


\colorlet{light-red}{red!20!white}
\colorlet{very-light-red}{red!5!white}
%\colorlet{dark-red}{red!50!black!20}
\definecolor{dark-red}{HTML}{8D0011}
\definecolor{dark-orange}{HTML}{FC9F3E}

\colorlet{light-blue}{blue!20!white}
\colorlet{very-light-blue}{blue!5!white}
\colorlet{dark-blue-50}{blue!50}
%\colorlet{dark-blue}{blue!50!black!20}
\definecolor{dark-blue}{HTML}{34AACF}


\pgfdeclarelayer{background}
\pgfsetlayers{background,main}

%=======================================================
%
% 	Boxes used for graphs
%
%----------------------------------

\tikzset{bluebox/.style={
			  rectangle,
			  drop shadow,
			  rounded corners = 1mm,
			  minimum width   = 20mm,
			  minimum height  = 10mm,
			  very thick, 
			  draw            = blue!60!black!40,
			  top color       = white,
			  bottom color    = blue!50!black!20,
			  %font            = \itshape,
			  text centered,
			  text width      = 40mm
			}}
\tikzset{greenbox/.style={
			  rectangle,
			  drop shadow,
			  rounded corners = 1mm,
			  minimum width   = 20mm,
			  minimum height  = 10mm,
			  very thick, 
			  draw            = green!60!black!40,
			  top color       = white,
			  bottom color    = green!50!black!20,
			  %font            = \itshape,
			  text centered,
			  text width      = 40mm
			}}			
\tikzset{textfeld/.style={
			  chamfered rectangle,
			  chamfered rectangle corners = north west,
			  drop shadow,
			  very thick,
			  draw            = yellow!50!black!50,
			  top color       = white,
			  bottom color    = yellow!50!black!20,
			  %font            = \itshape,
			  minimum width   = 10mm,
			  minimum height  = 15mm,
			  text centered,
			  text width      = 20mm
}}
\tikzset{redbox/.style={
			  rectangle,
			  drop shadow,
			  rounded corners = 1mm,
			  minimum width   = 25mm,
			  minimum height  = 10mm,
			  very thick, 
			  draw            = red!60!black!40,
			  top color       = white,
			  bottom color    = red!50!black!20,
			  %font            = \itshape,
			  text centered,
			  text width      = 50mm
			}}			

\tikzset{branch/.style={
			  diamond, 
			  drop shadow,
			  shape aspect    = 1.4,
			  minimum height  = 10mm,
			  minimum width   = 10mm,
			  text centered,
			  %text width      = 20mm,
			  very thick, 
			  draw            = red!50!black!50,
			  top color       = white,
			  bottom color    = red!50!black!20,
			  font            = \itshape
			}}
\tikzset{speicher-wo-shadow/.style={
			  cylinder,
			  shape border rotate = 90,
			  aspect	  = .10,
			  minimum width   = 22mm,
			  text centered,
			  text width      = 30mm,
			  very thick,
			  draw		  = yellow!50!black!50,
			  top color       = white,
			  bottom color    = yellow!50!black!20
			}}
\tikzset{speicher/.style={
			  speicher-wo-shadow,
			  drop shadow
			}}
\tikzset{information text/.style={
			  rounded corners,
			  fill=gray90, 
			  inner sep=1ex,
			  drop shadow
			}}
			

%=======================================================
%
% 	Cuboids
%
%----------------------------------

\newif\ifcuboidshade
\newif\ifcuboidemphedge

\tikzset{
  cuboid/.is family,
  cuboid,
  shiftx/.initial=0,
  shifty/.initial=0,
  dimx/.initial=3,
  dimy/.initial=3,
  dimz/.initial=3,
  scale/.initial=1,
  densityx/.initial=1,
  densityy/.initial=1,
  densityz/.initial=1,
  rotation/.initial=0,
  anglex/.initial=0,
  angley/.initial=90,
  anglez/.initial=225,
  scalex/.initial=1,
  scaley/.initial=1,
  scalez/.initial=0.5,
  front/.style={draw=black,fill=white},
  top/.style={draw=black,fill=white},
  right/.style={draw=black,fill=white},
  shade/.is if=cuboidshade,
  shadecolordark/.initial=black,
  shadecolorlight/.initial=white,
  shadeopacity/.initial=0.15,
  shadesamples/.initial=16,
  emphedge/.is if=cuboidemphedge,
  emphstyle/.style={thick},
}

\newcommand{\tikzcuboidkey}[1]{\pgfkeysvalueof{/tikz/cuboid/#1}}

% Commands
\newcommand{\tikzcuboid}[1]{
    \tikzset{cuboid,#1} % Process Keys passed to command
  \pgfmathsetlengthmacro{\vectorxx}{\tikzcuboidkey{scalex}*cos(\tikzcuboidkey{anglex})*28.452756}
  \pgfmathsetlengthmacro{\vectorxy}{\tikzcuboidkey{scalex}*sin(\tikzcuboidkey{anglex})*28.452756}
  \pgfmathsetlengthmacro{\vectoryx}{\tikzcuboidkey{scaley}*cos(\tikzcuboidkey{angley})*28.452756}
  \pgfmathsetlengthmacro{\vectoryy}{\tikzcuboidkey{scaley}*sin(\tikzcuboidkey{angley})*28.452756}
  \pgfmathsetlengthmacro{\vectorzx}{\tikzcuboidkey{scalez}*cos(\tikzcuboidkey{anglez})*28.452756}
  \pgfmathsetlengthmacro{\vectorzy}{\tikzcuboidkey{scalez}*sin(\tikzcuboidkey{anglez})*28.452756}
  \begin{scope}[xshift=\tikzcuboidkey{shiftx}, yshift=\tikzcuboidkey{shifty}, scale=\tikzcuboidkey{scale}, rotate=\tikzcuboidkey{rotation}, 
x={(\vectorxx,\vectorxy)}, y={(\vectoryx,\vectoryy)}, z={(\vectorzx,\vectorzy)}]
    \pgfmathsetmacro{\steppingx}{1/\tikzcuboidkey{densityx}}
  \pgfmathsetmacro{\steppingy}{1/\tikzcuboidkey{densityy}}
  \pgfmathsetmacro{\steppingz}{1/\tikzcuboidkey{densityz}}
  \newcommand{\dimx}{\tikzcuboidkey{dimx}}
  \newcommand{\dimy}{\tikzcuboidkey{dimy}}
  \newcommand{\dimz}{\tikzcuboidkey{dimz}}
  \pgfmathsetmacro{\secondx}{2*\steppingx}
  \pgfmathsetmacro{\secondy}{2*\steppingy}
  \pgfmathsetmacro{\secondz}{2*\steppingz}
  \foreach \x in {\steppingx,\secondx,...,\dimx}
  { \foreach \y in {\steppingy,\secondy,...,\dimy}
    {   \pgfmathsetmacro{\lowx}{(\x-\steppingx)}
      \pgfmathsetmacro{\lowy}{(\y-\steppingy)}
      \filldraw[cuboid/front] (\lowx,\lowy,\dimz) -- (\lowx,\y,\dimz) -- (\x,\y,\dimz) -- (\x,\lowy,\dimz) -- cycle;
    }
    }
  \foreach \x in {\steppingx,\secondx,...,\dimx}
  { \foreach \z in {\steppingz,\secondz,...,\dimz}
    {   \pgfmathsetmacro{\lowx}{(\x-\steppingx)}
      \pgfmathsetmacro{\lowz}{(\z-\steppingz)}
      \filldraw[cuboid/top] (\lowx,\dimy,\lowz) -- (\lowx,\dimy,\z) -- (\x,\dimy,\z) -- (\x,\dimy,\lowz) -- cycle;
        }
    }
    \foreach \y in {\steppingy,\secondy,...,\dimy}
  { \foreach \z in {\steppingz,\secondz,...,\dimz}
    {   \pgfmathsetmacro{\lowy}{(\y-\steppingy)}
      \pgfmathsetmacro{\lowz}{(\z-\steppingz)}
      \filldraw[cuboid/right] (\dimx,\lowy,\lowz) -- (\dimx,\lowy,\z) -- (\dimx,\y,\z) -- (\dimx,\y,\lowz) -- cycle;
    }
  }
  \ifcuboidemphedge
    \draw[cuboid/emphstyle] (0,\dimy,0) -- (\dimx,\dimy,0) -- (\dimx,\dimy,\dimz) -- (0,\dimy,\dimz) -- cycle;%
    \draw[cuboid/emphstyle] (0,\dimy,\dimz) -- (0,0,\dimz) -- (\dimx,0,\dimz) -- (\dimx,\dimy,\dimz);%
    \draw[cuboid/emphstyle] (\dimx,\dimy,0) -- (\dimx,0,0) -- (\dimx,0,\dimz);%
    \fi

    \ifcuboidshade
    \pgfmathsetmacro{\cstepx}{\dimx/\tikzcuboidkey{shadesamples}}
    \pgfmathsetmacro{\cstepy}{\dimy/\tikzcuboidkey{shadesamples}}
    \pgfmathsetmacro{\cstepz}{\dimz/\tikzcuboidkey{shadesamples}}
    \foreach \s in {1,...,\tikzcuboidkey{shadesamples}}
    {   \pgfmathsetmacro{\lows}{\s-1}
        \pgfmathsetmacro{\cpercent}{(\lows)/(\tikzcuboidkey{shadesamples}-1)*100}
        \fill[opacity=\tikzcuboidkey{shadeopacity},color=\tikzcuboidkey{shadecolorlight}!\cpercent!\tikzcuboidkey{shadecolordark}] (0,\s*\cstepy,\dimz) -- 
(\s*\cstepx,\s*\cstepy,\dimz) -- (\s*\cstepx,0,\dimz) -- (\lows*\cstepx,0,\dimz) -- (\lows*\cstepx,\lows*\cstepy,\dimz) -- (0,\lows*\cstepy,\dimz) -- cycle;
        \fill[opacity=\tikzcuboidkey{shadeopacity},color=\tikzcuboidkey{shadecolorlight}!\cpercent!\tikzcuboidkey{shadecolordark}] (0,\dimy,\s*\cstepz) -- 
(\s*\cstepx,\dimy,\s*\cstepz) -- (\s*\cstepx,\dimy,0) -- (\lows*\cstepx,\dimy,0) -- (\lows*\cstepx,\dimy,\lows*\cstepz) -- (0,\dimy,\lows*\cstepz) -- cycle;
        \fill[opacity=\tikzcuboidkey{shadeopacity},color=\tikzcuboidkey{shadecolorlight}!\cpercent!\tikzcuboidkey{shadecolordark}] (\dimx,0,\s*\cstepz) -- 
(\dimx,\s*\cstepy,\s*\cstepz) -- (\dimx,\s*\cstepy,0) -- (\dimx,\lows*\cstepy,0) -- (\dimx,\lows*\cstepy,\lows*\cstepz) -- (\dimx,0,\lows*\cstepz) -- cycle;
    }
    \fi 

  \end{scope}
}

%\makeatother

\newcommand*{\ShowIntersections}{
\fill 
    [name intersections={of=curve1 and curve2, name=i, total=\t}] 
    [ilr-blue, opacity=1, every node/.style={}] 
    \foreach \s in {1,...,\t}{(i-\s) circle (2pt)
        node {}};
}
