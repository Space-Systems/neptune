%------------------------------------------------------------------------------
\section{Satellite geometry model}
\label{sec:propagation-satellite-model}
%------------------------------------------------------------------------------

For the maintenance of an object catalogue in a space surveillance context, information about the shape of individual objects is, in general, limited or 
even unavailable. However, for an accurate modelling of non-gravitational forces, detailed knowledge of the surface properties is inevitable. This problem 
is typically dealt with by modelling the object as a point mass (or sphere of uniform density) and applying estimated values for the associated quantities in the 
non-gravitational model equations, like the drag parameter or the \gls{acr:srp} coefficient.

That design approach was also selected for \neptune, being a 3-\gls{acr:dof} propagator to be used in a space surveillance and tracking context. 
In contrast, a 6-\gls{acr:dof} propagation would not only require additional torque modelling, but also a detailed 3D modelling of the surfaces of the object under
consideration, which is the preferred solution for individual satellites, when high precision ephemerides (\acrshort{acr:poe}) with uncertainties in the cm-regime are to be
obtained. 

The validation process of \neptune (\cha{cha:validation}) is based on comparing computed ephemerides to available \gls{acr:poe} for distinct satellites and epochs. While this 
approach already showed quite good results, a further improvement to augment the 3-\gls{acr:dof} propagation is to apply simple \textit{box-wing-models} for some of those
satellites. It is thus possible to define individual surfaces for the \neptune propagation, with the following properties:
\begin{itemize}
 \item Surface area / \si{\metre\squared},
 \item specular and diffuse reflectivity coefficients as well as
 \item surface normal orientation. 
\end{itemize}

The orientation of the surface normal can be described by two angles for the initial state. For satellites, that are not in a randomly tumbling mode, one can 
also provide a model for the orientation during the propagation. Three different modes have been defined for \neptune with their associated angles to give 
the orientation, as shown in \tab{tab:sat-model-orientation}.
\begin{table}[h!]
 \centering
 \caption{Orientation modes and the associated defining angles for \neptune box-wing model propagation.\label{tab:sat-model-orientation}}
 \begin{tabular}{p{3cm}p{6cm}p{6cm}}
  \toprule
  Orientation mode & First angle & Second angle \\
  \midrule
  Earth-oriented   & Azimuth in satellite-centered horizontal plane (\gls{acr:ocrf}) & Elevation in satellite-centered horizontal frame (\gls{acr:ocrf}). \\
  \midrule
  Sun-oriented     & Offset to the right ascension of the Sun in the equatorial plane (\gls{acr:gcrf}) & Offset to the declination of the Sun perpendicular to the equatorial plane
\gls{acr:gcrf}. \\
  \midrule
  Inertially fixed & Right ascension (\gls{acr:gcrf}) & Declination (\gls{acr:gcrf}) \\ 
  \bottomrule
 \end{tabular}
\end{table}
Each time, the cross-section of the satellite is requested by the force model, \neptune will evaluate the current orientation based on the orientation mode 
and compute the value for that step. This might be, for example, the cross-section in relative velocity direction for the evaluation of the drag acceleration. 
In that case, the inner product of the relative velocity vector and the surface normal vector is evaluated and summed for all surfaces where that product 
is greater than zero:
\begin{equation}
 \gls{sym:area}_{\gls{idx:cross}}\left(\gls{sym:t}\right) = \sum_{i=0}^{n_{\gls{idx:srf}}} \gls{sym:area}_{i} \gls{sym:n0}\left(\gls{sym:t}\right)
\gls{sym:velvec}_{\gls{idx:rel}}\left(\gls{sym:t}\right), \quad \forall \gls{sym:n0}\left(\gls{sym:t}\right)
\gls{sym:velvec}_{\gls{idx:rel}}\left(\gls{sym:t}\right) > 0. \label{eq:satellite-model-normal}
\end{equation}

Note that this simple approach does not account for shadowing effects of individual surfaces. Also, due to the approach defined by \eq{eq:satellite-model-normal}, each defined
surface is only accounted for its positive normal direction. For instance, if modelling a single flat plate, it is important to account for both, the front and the rear side, as
shown exemplarily for a solar array in \gls{acr:leo} in \fig{fig:sat-model-solar-array}.
\begin{figure}[h!]
 \centering
 \begin{tikzpicture}
    \begin{axis}[
       xlabel = Time / min,
       ylabel = Cross-section / m$^2$,
       legend entries={Front only (1 surface), Front and rear (2 surfaces)},
       legend style={at={(1.2,1.0)}, anchor=north west, draw=none},
       legend cell align = left
    ]
    
      \addplot[red, dashed, very thick] table {07-Tikz/data/satModelFrontOnly.dat};
      \addplot[blue, densely dotted, very thick] table {07-Tikz/data/satModelFrontRear.dat};
    \end{axis}
\end{tikzpicture}
 \caption{Example for a Sun-oriented solar array with dimensions \SI{1}{\metre}$\times$\SI{1}{\metre} in a circular \gls{acr:leo}. The
cross-section in relative velocity direction is shown for both, a single defined surface, as well as a combined front and rear surface.\label{fig:sat-model-solar-array}}
\end{figure}
As expected, for a single surface, the cross-section is zero for half an orbit, as the angle between the surface normal direction and the relative velocity vector is greater than 
\ang{90;;}. However, as a real flat plate would also experience drag on the rear side, the correct result is only obtained for modelling two surfaces with normal
directions being 
\ang{180;;} apart.

In \fig{fig:sat-model-box-wing} an example is shown for the difference between a box-wing and a \textit{cannon-ball} model, where the latter corresponds to a sphere with a
cross-section that provides a best fit to the results of the box-winged satellite.
\begin{figure}[h!]
 \centering
 \begin{tikzpicture}
    \begin{axis}[
       xlabel = Time / min,
       xmin = 0,
       xmax = 240,
       ylabel = Error / m,
       legend entries={Radial, Along-track, Cross-track},
       legend style={at={(1.2,1.0)}, anchor=north west, draw=none},
       legend cell align = left
    ]
      \addplot[red, dashed, very thick]          table {07-Tikz/data/comp3.out.radial};
      \addplot[blue, dotted, very thick] table {07-Tikz/data/comp3.out.along};
      \addplot[green, densely dotted, very thick] table {07-Tikz/data/comp3.out.cross};
    \end{axis}
\end{tikzpicture}
 \caption{Example for the difference between a box-wing and a cannon-ball model. An Earth-oriented cube (10 \si{\metre\squared} surface, each) with one Sun-oriented solar array
(10 \si{\metre\square} area) in a circular \gls{acr:leo}. The radial, along-track and cross-track components are shown.\label{fig:sat-model-box-wing}}
\end{figure}
It can be seen, that radial and along-track periodic variations are introduced, which are in the range of a few metres. A box-wing model can thus be used for improved comparisons
in the validation process of \neptune (\cha{cha:validation})\todo[Validation]{Use box-wing model for validation of NEPTUNE with POE.}.

In the example shown in \fig{fig:sat-model-box-wing}, two different orientations for the surfaces of the box-wing model were combined: an Earth-oriented cube and a Sun-oriented
solar array. In the current implementation, \neptune allows for an easy computation of a \textit{minimum solar incidence angle}, which is the minimum angle between the surface normal and the Sun
direction for a given orbit. For example, a geostationary orbit would experience a variation of the minimum incidence angle between \ang{0;;} and \ang{23.5;;} over the course of a
year due to its orientation with respect to the ecliptic.




